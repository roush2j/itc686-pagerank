\section{History}

PageRank was developed at Stanford University by Larry Page and Sergey Brin in 
1996 as part of a research project to create a new search engine. 
The first paper about PageRank and the initial prototype implementation were 
published in 1998. 
The algorithm was and remains the core the Google Search service.

PageRank was heavily influenced by citation analysis developed by Eugene 
Garfield in the 1950s, and by HyperSearch developed by Massimo Marchiori.
Further influence came from Jon Kleinberg, who published his 
work on HITS (Hyperlink-Induced Topic Search) in the same year \cite{pagerankwiki}.

The name is a play on both Page's name and the web-\emph{pages} on which
the algorithm operates.
The word is trademarked by Google but the patent was assigned to Stanford 
University, with Google holding exclusive rights to exploitation. 
The University received 1.8 million shares in exchange for the patent, which 
it cashed out in 2005 for \$336 million. 


